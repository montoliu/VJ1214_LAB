\chapterimage{chapter_head_1.pdf} 
\chapter{El modo framebuffer}

Este capítulo explica el funcionamiento del modo framebuffer. La lista de ejercicios y el tiempo estimado (en minutos) para su realización se muestran en la Tabla \ref{c6_tab:ejercios}.

\begin{table}[t]
	\centering
	\caption{Ejercicios del capítulo y tiempo estimado para su realización.}
	\begin{tabular}{|c|c|}
		\hline 
		Ejercicio & Tiempo \\ 
		\hline 
		6.1 & 20' \\ 
		6.2 & 15' \\ 
		6.3 & 25' \\ 
		6.4 & 15' \\ 
		6.5 & 15' \\ 
		6.6 & 30' \\ 
		\hline 
	\end{tabular} 
	\label{c6_tab:ejercios}
\end{table}

% -----------------------------------------------------
% -----------------------------------------------------
% -----------------------------------------------------
% -----------------------------------------------------
\section{El modo Framebuffer}
La principal peculiaridad de este modo de representación es que la pantalla se mapea directamente a la memoria, sin utilizar el motor de renderizado, de forma que lo que se escribe en memoria es lo que se muestra. Además el modo Framebuffer solo está disponible para el motor principal. 

La pantalla se compone de $49\ 152$ píxeles, organizados en 192 líneas de 256 píxeles cada una, donde a su vez, el contenido de cada píxel representa un color en formato RGB de 16 bits.

El formato RGB de 16 bits para especificar el color se representa como una mezcla de una cantidad determinada del color rojo (R), verde (G) y azul (B), dedicando 5 bits para determinar esa cantidad: 0 indica ausencia de color y el 31 el máximo color. La librería \textit{libnds} proporciona la macro \textit{RBG15} para facilitar la definición de colores. Basta con indicar la cantidad de cada uno de ellos, en un rango de 0 a 31, para obtener el color en formato RGB de 16 bits. Por ejemplo, para obtener los colores: rojo, verde, azul y negro, habrá que usar las macros: \textit{RGB15(31,0,0)}, \textit{RGB15(0,31,0)}, \textit{RGB15(0,0,31)}, \textit{RGB15(0,0,0)}, respectivamente. El uso de esta macro simplifica bastante la tarea de dibujado en la pantalla, ya que únicamente hay que asignar píxel a píxel el valor correspondiente al color que
se desea pintar.

Tal como muestra la Figura \ref{fig_c5_modulos} únicamente el motor principal puede usar el modo framebuffer. Esto implica que solo se podrá mostrar una imagen en una de las dos pantallas a la vez.
% -----------------------------------------------------
% -----------------------------------------------------
\section{Preparando el modo framebuffer}
El modo framebuffer admite cuatro tipos de configuración, que reciben el nombre de FB0, FB1, FB2 o FB3 y que básicamente establecen la zona de memoria cuyo contenido se va a volcar en la pantalla. Las zonas de memoria donde se deberá volcar la información en cada una de las cuatro configuraciones son VRAM\_A, VRAM\_B, VRAM\_C o VRAM\_D, respectivamente.

En primer lugar hay que especificar el modo seleccionado, para lo cual se utiliza el registro REG\_DISPCNT y la macro que se corresponda con el modo a utilizar. Por ejemplo, para usar la configuración FB0 (y por lo tanto la zona de memoria VRAM\_A) se deberá usar el siguiente código:

\begin{verbatim}
REG_DISPCNT = MODE_FB0;
\end{verbatim}

Al especificar el modo de vídeo, indirectamente se está especificando la zona de memoria donde se deberán escribir los datos que se van a mostrar en la pantalla. Para poder utilizar los bancos de memoria es necesario que éstos estén habilitados y configurados según su finalidad. Para ello hay que usar el siguiente código:

\begin{verbatim}
VRAM_A_CR = VRAM_ENABLE | VRAM_A_LCD;
\end{verbatim}

Una vez especificado el modo y habilitada la zona de memoria donde se escribirán los datos, el siguiente paso consiste en escribir en la memoria el valor que se asignará a cada uno de los píxeles de la pantalla, bien mediante una asignación manual, utilizando la macro RGB15 o escribiendo los datos correspondientes a una imagen específica.

% -----------------------------------------------------
% -----------------------------------------------------
\section{Mostrar imágenes}
\label{sec:p2_c3_imagenes}
El modo framebuffer no utiliza fondos ni motor de renderizado, por lo que la única tarea a realizar para mostrar una imagen es transferir al banco de memoria correspondiente los datos del gráfico a mostrar. Para ello, será necesario convertir esa imagen en un mapa de bits.

La conversión de imágenes se realiza con la herramienta \textit{grit}, que ya viene integrada en \textit{devkitPro}. En concreto, el comando \textit{grit} se encuentra en el directorio \textit{devkitPro/tools/bin/grit.exe} \footnote{En versiones anteriores se encontraba en \textit{devkitPro/devkitARM/bin/grit.exe}}. Por ejemplo, para convertir la imagen \textit{imagen.png} al formato que se necesita, se deberá ejecutar la siguiente orden:

\begin{verbatim}
grit imagen.png -gb -gB16
\end{verbatim}

La primera opción (-gb) indica que el formato de la conversión es un mapa de bits y la segunda opción (-gB16), que tiene una profundidad de 16 bits por píxel.

El comando creará un fichero de cabecera \textit{imagen.h} y un fichero \textit{imagen.s} que contiene el vector de datos correspondiente al mapa de bits de la imagen especificada (en lenguaje ensamblador ARM). El fichero de cabecera deberá ser incluido en el programa principal con la orden:

\begin{verbatim}
#include "imagen.h"
\end{verbatim}
	
El fichero \textit{imagen.s} deberá ser añadido al proyecto para que el linker puede añadir la imagen al ejecutable.

\begin{example}
El siguiente programa (\textit{unaimagen.c}) muestra la imagen \textit{mario.png} en la pantalla inferior de la NDS:

\begin{lstlisting}
#include <nds.h>
#include "mario.h"

int main (void)
{
	REG_POWERCNT = POWER_LCD | POWER_2D_A;
	REG_DISPCNT  = MODE_FB0;
	VRAM_A_CR    = VRAM_ENABLE | VRAM_A_LCD;
	dmaCopy(marioBitmap, VRAM_A, 256*192*2);
	while (1)
	{
		swiWaitForVBlank();
	}
	return 0;
}
\end{lstlisting}
\end{example}

La función \textit{dmaCopy} se encarga de copiar el vector de datos \textit{marioBitmap} (declarado en mario.h) al banco de memoria (VRAM\_A) que está mapeado a la pantalla.

\begin{exercise}
Crea un nuevo proyecto, usando como punto de partida el programa \textit{unaimagen.c}, que muestre en la pantalla inferior una imagen de 256*192 píxeles. Recuerda que debes obtener los ficheros \textit{.h} y \textit{.c} de la imagen usando el comando \textit{grit} y que dichos ficheros deben estar incluidos en el proyecto.
\end{exercise}


\begin{example}
En el siguiente programa (\textit{dosimagenes.c}) se cargan dos imágenes (una en VRAM\_A y la otra en VRAM\_B). Cuando se pulsa la tecla A (de la NDS) se mostrará una de ellas, cuando se pulsa la tecla X (de la NDS) se mostrará la otra:

\begin{lstlisting}
#include <nds.h>
#include "mario.h"
#include "wallpaper.h"

int main (void)
{
	REG_POWERCNT = POWER_LCD | POWER_2D_A;
	
	VRAM_A_CR    = VRAM_ENABLE | VRAM_A_LCD;
	dmaCopy(marioBitmap, VRAM_A, 256*192*2);
	
	VRAM_B_CR    = VRAM_ENABLE | VRAM_B_LCD;
	dmaCopy(wallpaperBitmap, VRAM_B, 256*192*2);
	
	// De inicio se muestra la que se encuentra en VRAM_A (FB0)
	REG_DISPCNT  = MODE_FB0;
	while (1)
	{
		scanKeys();
		int held=keysUp();
		
		if (held & KEY_A)
		{
			REG_DISPCNT  = MODE_FB0;
		}
		if (held & KEY_X)
		{
			REG_DISPCNT  = MODE_FB1;
		}
		
		swiWaitForVBlank();
	}
	return 0;
}
\end{lstlisting}
\end{example}

Como se puede comprobar, en primer lugar se carga la primera imagen en VRAM\_A y la segunda en VRAM\_B. Para seleccionar que imagen se quiere mostrar, es necesario especificarlo cambiando el contenido del registro REG\_DISPCNT. Los ficheros \textit{.s} de ambas imágenes deben estar incluidos en el proyecto.

\begin{exercise}
Crea un nuevo proyecto, usando como punto de partida el programa \textit{dosimagenes.c}, que muestre en la pantalla inferior una imagen de 256*192 píxeles. Cuando se pulse la tecla X se cambiará la imagen por otra. Si se pulsa de nuevo la tecla X, se volverá a cambiar la imagen mostrada y así sucesivamente. En total existirán 4 imágenes diferentes.
\end{exercise}

\begin{exercise}
Modifica el programa anterior para que el cambio de imagen a mostrar se realice cada un número determinado de segundos (por ejemplo cada 2 segundos). Al final del capítulo 3 hay un ejemplo que te puede servir de mucha ayuda.
\end{exercise}
	
% -----------------------------------------------------
% -----------------------------------------------------
\section{Dibujar píxeles}
\label{sec:p2_c3_pixeles}
Otra forma alternativa de representar gráficos en modo framebuffer consiste en componer una imagen a partir de la asignación de colores a cada uno de los píxeles que componen la pantalla. Al igual que en el caso anterior, el primer paso consistirá en especificar el modo de vídeo utilizado. 
	
El siguiente paso difiere del apartado anterior, donde se hacía una transferencia directa de un vector de datos, correspondiente a la imagen a dibujar. En este caso, se accederá directamente a la región de memoria VRAM utilizada, para escribir uno por uno el valor del color que se desee asignar a cada uno de los píxeles que componen la pantalla. La macro \textit{RGB15} permite obtener ese valor, indicando la cantidad de color rojo, verde y azul de la que se compone el color que se pintará en cada píxel.

\begin{example}
Supongamos que se desea mostar un degradado vertical de color negro a azul, que ocupe toda la pantalla. El color negro se obtiene mediante la ausencia de color de las tres tonalidades, por lo tanto, mediante \textit{RGB15(0,0,0)} obtenemos el valor correspondiente. Para hacer el degradado se deberá ir incrementando la cantidad de azul a medida que se vaya descendiendo en la pantalla. Dicho de otra forma, el nivel de color azul depende exclusivamente de la línea que estemos pintando. Por tanto basta con recorrer todos los puntos de la pantalla y pintarlos con el color correspondiente a la línea. El programa (\textit{degradado.c}) se podría implementar como sigue:
	
\begin{lstlisting}	
#include <nds.h>
#include <stdio.h>

int main (void)
{
	REG_POWERCNT = POWER_LCD | POWER_2D_A;
	REG_DISPCNT  = MODE_FB0;
	VRAM_A_CR    = VRAM_ENABLE | VRAM_A_LCD;
		
	int lin, col;
	unsigned short *fb = VRAM_A;
	for(lin=0; lin<192; lin ++)
		for(col=0; col<256; col ++)
			fb[lin*256 + col] = RGB15 (0, 0, lin*32/192);
		
	while(1)
	{
		swiWaitForVBlank();
	}
	return 0;
}	
\end{lstlisting}
\end{example}
	
En la línea 11 se define un puntero de tipo entero sin signo de 16 bits y se hace que éste apunte a la dirección 0x6800000 (valor de la macro VRAM\_A), que es el comienzo de la memoria del framebuffer FB0. El hecho de definir el puntero a la memoria como \textit{unsigned short} es porque en este modo se necesita direccionar la memoria píxel a píxel de la pantalla que, como ya se ha comentado anteriormente, ocupan cada uno 16 bits (5 bits por componente y el bit más significativo sin usar).

En la línea 14, se utiliza el puntero fb para acceder al píxel correspondiente a la línea \textit{lin} y a la columna \textit{col}. Dado que los píxeles se almacenan por líneas y cada línea tiene 256 puntos tendremos que multiplicar el número de línea por 256 para ir al comienzo de la fila y sumar el número de columna para llegar al píxel correspondiente de la pantalla. 


\begin{exercise}
Modifica el programa anterior para que el degradado sea horizontal en vez de vertical. Para hacer el degradado se deberá ir incrementando la cantidad de azul a medida que se vaya desplazando hacia la derecha en la pantalla. Es decir, el nivel de color azul depende exclusivamente de la columna que estemos pintando.
\end{exercise}

\begin{example}
El siguiente programa (\textit{colores.c}) explica como cambiar la imagen que se muestra en la pantalla. Al pulsar las teclas A, X y B (de la NDS) se mostrará la pantalla de color rojo, verde y azul, respectivamente. El código se muestra a continuación:

\begin{lstlisting}	
#include <nds.h>
#include <stdio.h>

int main (void)
{
	
	REG_POWERCNT = POWER_LCD | POWER_2D_A;
	VRAM_A_CR    = VRAM_ENABLE | VRAM_A_LCD;
	VRAM_B_CR    = VRAM_ENABLE | VRAM_B_LCD;
	VRAM_C_CR    = VRAM_ENABLE | VRAM_A_LCD;
	
	u16 color_rojo  = RGB15(31,0,0);
	u16 color_verde = RGB15(0,31,0);
	u16 color_azul  = RGB15(0,0,31);
	
	unsigned short *fbA = VRAM_A;
	unsigned short *fbB = VRAM_B;
	unsigned short *fbC = VRAM_C;
	
	int lin,col;
	
	for(lin = 0; lin < 192; lin ++)
		for(col = 0; col < 256; col ++)
			fbA[lin*256 + col] = color_rojo;
	
	for(lin = 0; lin < 192; lin ++)
		for(col = 0; col < 256; col ++)
			fbB[lin*256 + col] = color_verde;
	
	for(lin = 0; lin < 192; lin ++)
		for(col = 0; col < 256; col ++)
			fbC[lin*256 + col] = color_azul;
	
	// de inicio FB0
	REG_DISPCNT  = MODE_FB0;
	while (1)
	{
		scanKeys();
		int held=keysHeld();
		if (held & KEY_A)
		{
			REG_DISPCNT  = MODE_FB0;
		}
		if (held & KEY_X)
		{
			REG_DISPCNT  = MODE_FB1;
		}
		if (held & KEY_B)
		{
			REG_DISPCNT  = MODE_FB2;
		}
		swiWaitForVBlank();
	}
	return 0;
}
\end{lstlisting}
\end{example}


\begin{exercise}
Realiza un programa que al pulsar las teclas A, X y B (de la NDS) muestre las banderas de tres países diferentes.
\end{exercise}


% -----------------------------------------------------
% -----------------------------------------------------
\section{Ejercicios avanzados}

\begin{exercise}
Realiza un programa que permita mover un cuadrado de 25x25 píxeles por la pantalla. Para ello se usarán los botones de las flechas de la NDS. Se deberá tener en cuenta los límites de la pantalla.
\end{exercise}






